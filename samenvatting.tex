\documentclass[11pt]{article}
\usepackage[pdftex]{graphicx}
\usepackage[dutch]{babel}
\usepackage{amsthm, amsmath, enumitem, fullpage, helvet, color}
\usepackage[parfill]{parskip}
\let\originalitem\item
\renewcommand{\item}{\originalitem[]}
\renewcommand*{\familydefault}{\sfdefault}
\renewcommand{\d}{\textrm{d}\,}
\usepackage{mathtools}


\usepackage[active,tightpage]{preview}
\renewcommand{\PreviewBorder}{1in}
\newcommand{\Newpage}{\end{preview}\begin{preview}}

\title{Samenvatting elektronische signalen}
\author{Haroen Viaene}
\date{10 november 2014}


\begin{document}
\begin{preview}

\selectlanguage{dutch}

\maketitle

\tableofcontents

\Newpage

\section{Inleiding}

\begin{itemize}
\item Q
	\begin{itemize}
		\item{lading}
		\item Coulomb
		\item $6,35 * 10^{18}$ elektronen
		\item (veranderlijk? q)
	\end{itemize}
\item I
	\begin{itemize} 
		\item stroom
		\item Amp\`ere
		\item $\frac{Q}{t}$
		\item (veranderlijk? $\frac{\d q}{\d t}$)
		\item teken
			\begin{itemize}
				\item afhankelijk van stroomzin
				\item mee met stroomzin is positief
				\item conventioneel: + naar -
				\item elektronenstroom: - naar +
			\end{itemize}
	\end{itemize}
\item U
	\begin{itemize}
		\item potentiaalverschil
		\item Volt
		\item spanning
		\item $\frac{E}{Q}$
		\item positief geladen: positief potentiaal
	\end{itemize}
\item R
	\begin{itemize}
		\item weerstand
		\item $\Omega$
		\item doorgang van elektronische stroom belemmerend
		\item wet van Ohm
		\begin{itemize}
			\item $R=\frac{u}{i}$
		\end{itemize}
		\item wet van Pouillet
		\begin{itemize}
			\item $R = \rho * \frac{L}{A}$
			\item $\rho$ is resistiviteit
			\item A is doorsnede
			\item L is lengte
		\end{itemize}
	\end{itemize}
\item W
	\begin{itemize}
		\item arbeid
		\item Joule (Wattseconde)
		\item $W=U \cdot Q=U\cdot I \cdot t$
		\begin{itemize}
			\item maak ook gebruik van wet van Ohm
		\end{itemize}
	\end{itemize}
\item P
	\begin{itemize}
		\item vermogen
		\item Watt
		\item $P=\frac{W}{t}=\frac{U \cdot I \cdot t}{t} = U \cdot I$
		\item gebruik ook wet van Ohm 
	\end{itemize}
\end{itemize}

\Newpage
\section{weerstanden en 1 bron}

\begin{itemize}
\item serieschakeling
	\begin{itemize}
		\item pas wet van Ohm toe op elk van de weerstanden
		\begin{itemize}
			\item I constant
			\item $E=I \cdot R_v$
			\item Bewijs %Hoe te kennen?
			\item $R_v = \sum_i R_i$
		\end{itemize}
		\item spanningsdeler
	\end{itemize}
\item parallelschakeling
	\begin{itemize}
		\item pas wet van Ohm toe op elk van de weerstanden
		\begin{itemize}
			\item U constant
			\item $I = \frac{U}{\sum_i R_i}$
			\item $\frac{1}{R_v} = \sum_i \frac{1}{R_i}$
		\end{itemize}
		\item twee weerstanden? $R_v = \frac{R_1 \cdot R_2}{R_1 = R_2}$
		\item stroomdeler
		\item let op! vermogen! %wat?
	\end{itemize}
\item gemengde schakeling
	\begin{itemize}
		\item los op in deelstukjes
	\end{itemize}
\end{itemize}

\Newpage
\section{Bronnen}

\begin{itemize}
\item spanningsbron
	\begin{itemize}
		\item constante spanning, onafhankelijk stroom
		\item inwendige weerstand in serie
			\begin{itemize}
				\item $U = E - R_i \cdot I$
			\end{itemize}
		\item randvoorwaarden
		\begin{itemize}
			\item $I = 0 \rightarrow U = E$
			\item $U = 0 \rightarrow I = \frac{E}{R_i}$
		\end{itemize}
		\item belastingslijnen \textbf{huh? Wat is de boedoeling van die lijnen?}
	\end{itemize}
\item stroombron
	\begin{itemize}
		\item constante stroom, onafhankelijk van spanning
		\item inwendige weerstand in parallel
		\begin{itemize}
			\item $I = 0 \rightarrow U = R_i \cdot I_0$
			\item $U = 0 \rightarrow I = I_{R_b}$
		\end{itemize}
		\item Als $R_i$ constant is:
		\begin{itemize}
			\item $R_i = \frac{-I}{U}$
		\end{itemize}
	\end{itemize}
\end{itemize}

\Newpage
\section{Meerdere bronnen}

\textit{Hier heb ik nog wat moeite mee, kan iemand dat eens in mensentermen uitleggen?}

\begin{itemize}
	\item Kirchoff
	\begin{itemize}
		\item De som van voltages in een gesloten kring moet 0 zijn.
		\item De som van voltages op een knooppunt moet 0 zijn.
	\end{itemize}
	\item Superpositie
	\begin{itemize}
		\item Bekijk eerst 1 van de bronnen en laat de andere weg, dan omgekeerd. Tel op.
	\end{itemize}
	\item Th\'evenin
	\begin{itemize}
		\item Vervang het circuit door 1 weerstand en 1 bron
	\end{itemize}
\end{itemize}

\Newpage
\section{Weerstanden}

\begin{itemize}
	\item begrenst stroom
	\item tolerantie
	\begin{itemize}
		\item zegt hoe veel verschil er op de verwachte waarde kan zijn (in procent)
		\item vb: $100 \Omega, 5\% \rightarrow 100 \Omega \pm 5 \Omega (95 \Omega \textrm{ tot } 105 \Omega)$
	\end{itemize}
	\item dissipatievermogen
	\begin{itemize}
		\item vermogen dat verloren kan gaan aan warmte
		\item als je boven dat vermogen gaat, komt er schade
		\item $P = I^2 \cdot R = \frac{U^2}{R}$
		\item maximale stroom: $I_{\textrm{max}} = \sqrt{\frac{P_d}{R}}$ (dankzij: $P=R \cdot I^2$)
	\end{itemize}
	\item lineaire weerstanden
	\begin{itemize}
		\item draadgewonden: draaad rond koper/ijzer/nikkellegering, glazuur afdeklaag (soms keramiek)
		\item koollaag: laag koolstof
		\item metaalfilm: laagje metaal
		\item SMD: surface mounting devices (passen beter op PCB)
	\end{itemize}
	\item niet-lineaire weerstanden
	\begin{itemize}
		\item NTC: weerstand omgekeerd evenredig met temperatuur: onnauwkeurige temperatuurmeting
		\item PTC: weerstand evenredig met temperatuur: gebruikt tegen oververhitting
		\item LDR: lichtgevoelig, traag (50-200 ms)
		\item VDR: minder weerstand als meer spanning
	\end{itemize}
	\item regelbare weerstanden
	\begin{itemize}
		\item potentiometer
		\item uiterste klemmen: vaste waarde, middelste en buitenste: regelbaar
		\item te zien als een weerstand die je opdeelt
		\item lineair of logaritmisch
	\end{itemize}
\end{itemize}

\Newpage
\section{Condensatoren}

\begin{itemize}
	\item laat wisselspanning door, houdt gelijkspanning tegen
	\item 2 geleiders gescheiden door een isolator (di\"electricum)
	\item parameters
	\begin{itemize}
		\item capaciteit
		\begin{itemize}
			\item $C = \frac{\epsilon \cdot A}{d} $ ($\epsilon$ is de di\"electrische constante)
			\item $C = \frac{\Delta Q}{\Delta V} $
			\item soms worded de platen ge\"etst om meer oppervlak te hebben (elektrolytishce condensatoren)
		\end{itemize} 
		\item werkspanning %%TODO: verstaan
		\begin{itemize}
			\item als spanning te hoog is, komt er geleiding over de platen, en brandt het di\"electricum door.
		\end{itemize}
		\item Tolerantie
	\end{itemize}
	\item spanning over condensator
	\begin{itemize}
		\item $C = \frac{q}{u} = \frac{I \cdot t }{u} $
	\end{itemize}
	\item variabele stroom
	\begin{itemize}
		\item $C \cdot \frac{\Delta V}{\Delta t} = i $
		\item stroom is de capaciteit $\cdot$ de afgeleide van het potentiaalverschil over de tijd
	\end{itemize}
	\item regelbare condensatoren
	\begin{itemize}
		\item afstand tussen platen wijzigbaar
		\item di\"electricum meestal lucht
		\item Capaciteit regelbaar
	\end{itemize}
	\item serie
	\begin{itemize}
		\item Q is constant
		\item $\frac{1}{C_v} = \sum_i \frac{1}{\frac{1}{C_i}} $
		\item maximale spanning wordt opgeteld
	\end{itemize}
	\item parallel
	\begin{itemize}
		\item $C_v = \sum_i C_i $
	\end{itemize}
\end{itemize}

\Newpage
\section{Spoelen}

\begin{itemize}
	\item grote weerstand aan wisselstroom, kleine aan gelijkstroom
	\item afvlakkende pulserende gelijksstroom
	\item magneten
	\item filters
	\begin{itemize}
		\item Inductiviteit
		\begin{itemize}
			\item H
		\end{itemize}
		\item Werkspanning
		\begin{itemize}
			\item V
		\end{itemize}
	\end{itemize}
\end{itemize}

\end{preview}
\end{document}